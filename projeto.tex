\documentclass[a4paper, 11pt]{article}

\usepackage[brazilian]{babel}
\usepackage[utf8]{inputenc}
\usepackage[T1]{fontenc}
\usepackage{array}
\usepackage{fixltx2e}
\usepackage{amsmath}
\usepackage{amssymb}
\usepackage{graphicx}
\usepackage{float}
\usepackage{a4wide}
\usepackage{multicol}
\usepackage[table]{xcolor}
\usepackage{makeidx}
\usepackage{hyperref}
\usepackage{setspace}
\usepackage{indentfirst}
\usepackage[nottoc]{tocbibind}
\usepackage{listings}
\usepackage[a4paper,top=2.54cm,bottom=2.0cm,left=2.0cm,right=2.54cm]{geometry}
\usepackage[square, sort, comma, numbers]{natbib}

\onehalfspacing
\graphicspath{{./img/}}
\makeindex

\usepackage{tikz}
\usetikzlibrary{calc,trees,positioning,arrows,chains,shapes.geometric,%
    decorations.pathreplacing,decorations.pathmorphing,shapes,%
    matrix,shapes.symbols}

\tikzset{
>=stealth',
  punktchain/.style={
    rectangle, 
    rounded corners, 
    draw=black, very thick,
    text width=8.3em, 
    minimum height=3em, 
    text centered, 
    on chain},
  fillchain/.style={
    rectangle, 
    rounded corners, 
    fill=blue!30,
    draw=black, very thick,
    text width=8.3em, 
    minimum height=3em, 
    text centered, 
    on chain},
  line/.style={draw, thick, <-},
  element/.style={
    tape,
    top color=white,
    bottom color=blue!50!black!60!,
    minimum width=8em,
    draw=blue!40!black!90, very thick,
    text width=8.3em, 
    minimum height=3.5em, 
    text centered, 
    on chain},
  every join/.style={->, thick,shorten >=1pt},
  decoration={brace},
  tuborg/.style={decorate},
  tubnode/.style={midway, right=2pt},
}

\lstset{
    basicstyle=\footnotesize,
    morekeywords={either,or,transform,rule,to,from,through,function},
    frame=L,
}

\setcounter{secnumdepth}{2}
\setcounter{tocdepth}{2}

\begin{document}

\hypersetup{backref,pdfpagemode=FullScreen,colorlinks=true}

\thispagestyle{empty}
\begin{center}
    \vspace*{5cm}
    \textbf{\Large{Técnicas de Otimização Automática para a\\
            Portabilidade do Desempenho de Soluções para Problemas NP}}\\

    \vskip 1cm

    \textsc{Projeto de Pesquisa}

    \vskip 2cm

    \begin{minipage}{.4\linewidth}
        \begin{flushleft}
            \emph{Autor}: \\Pedro \textsc{Bruel}\\
            pedro.bruel@gmail.com
        \end{flushleft}
    \end{minipage}
    \begin{minipage}{.4\linewidth}
        \begin{flushright}
            \emph{Orientador}: \\Prof. Dr. Alfredo \textsc{Goldman}\\
            gold@ime.usp.br
        \end{flushright}
    \end{minipage}

    \vskip 1cm

    \normalsize{\emph{DCC - IME\\
    Universidade de São Paulo}\\}

    %\vskip 2cm
    %\begin{abstract}
    %\rule{\linewidth}{.1mm} \\
    %\rule{\linewidth}{.1mm} \\
    %\end{abstract}

    \vfill
    \normalsize{São Paulo, \today}
\end{center}

\newpage
\tableofcontents
\newpage

\begin{abstract}
\rule{\linewidth}{.1mm}
    
A grande variedade de arquiteturas de computador torna o desenvolvimento e a 
otimização de programas com eficiência portável um esforço que deve ser 
repetido para cada nova arquitetura. Trabalhos na área da otimização automática
de programas utilizam métodos empíricos, auxiliados pela crescente
disponibilidade de ciclos computacionais, para realizar uma busca por escolhas
de implementação e compilação que otimizem o desempenho da solução de um 
problema em diferentes arquiteturas de computador. O objetivo deste projeto de
pesquisa é utilizar as técnicas de Multi-Seleção e execução em algoritmo 
\emph{portfolio} --- como implementadas no sistema AMF --- para resolver 
problemas computacionalmente difíceis, no contexto do sistema PetaBricks,
que utiliza abstrações no nível de linguagem para especificar espaços de
soluções de um mesmo problema, e otimiza a solução para uma dada arquitetura
através de um processo de escolha algorítmica.

\rule{\linewidth}{.1mm}
\end{abstract}

\section{Introdução}

Recentes avanços na construção de arquiteturas de computador, como
por exemplo processadores com múltiplos núcleos e novas arquiteturas
e métodos de acesso a memória, criam novas possibilidades para o 
desenvolvimento de aplicações paralelas. Somando-se a esses avanços, o contínuo
barateamento e o aumento da disponibilidade de ciclos de processamento incitam 
a exploração de métodos empíricos e automáticos para a otimização de programas,
em busca da portabilidade não só de execução, mas também de desempenho. Este 
projeto de pesquisa relaciona alguns desenvolvimentos na área da otimização
automática de programas, e tem como objetivo combinar as estratégias utilizadas
nos sistemas PetaBricks (\citet{ansel2014phd}) e AMF 
(\citet{goldman2012framework}), para solução --- com eficiência portável entre
arquiteturas --- de problemas computacionalmente difíceis.

Encontram-se na literatura diversos de métodos e ferramentas para otimização 
automática de código que visam minimizar o esforço associado ao desenvolvimento
e manutenção de código portável, funcional e eficiente. Motivados pela grande
variedade e rápida mudança das arquiteturas de computador atuais, esses 
trabalhos procuram explorar de forma mais completa o paralelismo inerente a
elas, proveniente de múltiplos núcleos de processamento, 
diferentes arquiteturas de memória distribuída, compartilhada (de acesso 
uniforme, não uniforme), e auxílio de coprocessadores.

A otimização automática de um programa pode envolver modelagens da
arquitetura alvo, exigindo conhecimentos específicos sobre o problema e a
máquina --- ou buscas empíricas no espaço de possíveis soluções, encontrando
a melhor otimização através de técnicas de aprendizado de máquina e medidas de
desempenho.

No processo de busca empírica, alguns compiladores realizam uma calibragem dos 
parâmetros de compilação (\citet{bilmes1997,whaley1998}) para a arquitetura 
alvo, e utilizam essa calibragem para gerar código otimizado para essa 
arquitetura. 

Outros trabalhos (\citet{goldman2012framework,mitcsail-tr:2014,
vuduc2004}) propõe uma abordagem diferente, onde são geradas e executadas 
variadas combinações de algoritmos que resolvem um mesmo problema, comparando 
seus desempenhos para selecionar a melhor combinação para uma dada arquitetura.
Os enormes espaços de busca resultantes das diferentes formas de se implementar
um algoritmo --- ou combinações deles --- em determinada arquitetura, e a 
dificuldade de se modelar o impacto de pequenas variações na implementação 
desses algoritmos, também motivam o desenvolvimento de técnicas empíricas para
busca nesses espaços.

Nesta seção são discutidos o problema da otimização automática
(Seção \ref{sec:autotuning}) e os sistemas PetaBricks (Seção \ref{sec:peta}) e
AMF (Seção \ref{sec:amf}). O restante do projeto é organizado da seguinte 
forma: A Seção \ref{sec:trabrel} apresenta um levantamento bibliográfico com os
principais trabalhos relacionados à proposta do projeto.
A Seção \ref{sec:obj} apresenta os objetivos do projeto. Na Seção 
\ref{sec:met}, discute-se a metodologia de desenvolvimento e avaliação de 
resultados. Finalmente, a Seção \ref{sec:sched} apresenta um plano de trabalho
e um cronograma para seu desenvolvimento. 

\subsection{Autotuning} \label{sec:autotuning}

Ao desenvolver uma solução portável e eficiente para um problema, o
programador deve considerar, entre outros fatores, as possibilidades
de implementação dos algoritmos que resolvem o problema, e as otimizações e 
parâmetros disponibilizados pelo compilador.
A escolha da melhor implementação de um algoritmo e do melhor conjunto de 
valores para parâmetros de compilação depende em grande parte das arquiteturas 
e ambientes em que se pretende executar o programa, e o programador realiza 
essa escolha baseando-se em seus conhecimentos das arquiteturas alvo e 
resultados de testes que realiza durante o desenvolvimento.

No entanto, as arquiteturas de computador atuais variam muito, num grande 
espectro de, por exemplo, tamanhos de \emph{RAM},
velocidades e modos de acesso a níves de memória e armazenamento, número de 
núcleos no processador, e coprocessadores --- \emph{GPU}s, entre outros 
--- que apresentam ainda sua própria gama de variação.

Devido a essa heterogeneidade e sua rápida taxa de mudança, 
o trabalho de produzir implementações portáveis e eficientes 
dependerá cada vez mais de conhecimentos específicos sobre uma arquitetura,
motivando o desenvolvimento de técnicas automáticas de seleção das melhores
implementações para solução de um problema, isto é, o \emph{autotuning} de um
conjunto de soluções possíveis.

Como exemplo para a seleção de implementações diferentes para um mesmo
problema, considere a multiplicação de matrizes. O melhor algoritmo
para multiplicação de duas matrizes depende, por exemplo, do tamanho dessas
matrizes, do tempo de acesso e tamanho de cache do processador, e das 
possibilidades para exploração do paralelismo da arquitetura.
Para ilustrar a seleção de valores para parâmetros de compilação, considere
a escolha de opções, ou \emph{flags}, de compilação.
~\citet{tartara2013heuristics} apresentam um algoritmo para seleção de 
heurísticas de compilação, e utilizam a seleção de \emph{flags}
do GCC para demonstrar o algoritmo, implementado em extensões do PetaBricks e 
do próprio GCC. 
O alto número de \emph{flags} disponíveis numa dada arquitetura e as interações
complexas entre as diferentes opções fazem do conjunto dos problemas de 
seleção do melhor algoritmo para multiplicação de matrizes e de seleção de 
\emph{flags} de compilador, um bom candidato para solução por 
\emph{autotuning}.

O problema do \emph{autotuning}, através do processo de escolha algorítmica,
pode ser descrito da seguinte forma (\citet{ansel2011efficient}):

Considere o conjunto $A$ de algoritmos que resolvem instâncias de um dado 
problema, o conjunto $S$ de seletores para esses algoritmos, e o conjunto $S_t$
de seletores para parâmetros de compilação e configuração no conjunto $B$.
O seletor $s \in S$ de algoritmos em $A$ consiste em 
$\overrightarrow{C_s} = [c_{s,1},\dots,c_{s,m-1}] \cup \overrightarrow{A_s} = 
[\alpha_{s,1},\dots,\alpha_{s,m-1}]$,
onde cada $c_{s,i} \in \overrightarrow{C_s}$ representa um intervalo 
relacionado a um tamanho de entrada para a instância, e 
determina a execução do algoritmo $\alpha_{s,i} \in \overrightarrow{A_s}, 
\quad \alpha_{s,i} \in A$ neste intervalo. 
Além disso, $min(n) \leq c_{s,1} < n < c_{s,m-1} \leq max(n)$,
onde $n$ é o tamanho da entrada para a instância sendo selecionada, e $m$ 
é determinado pela instância.
Assim, um seletor $s \in S$ seleciona algoritmos da seguinte forma:

\begin{equation*}
    SELECT(s, i) = \alpha_{s,i} \in \overrightarrow{A_{s}} \quad | \quad c_{s,i} \in \overrightarrow{C_{s}},
    \quad\quad 1 \leq i < m
\end{equation*}

Além de um seletor para algoritmos, podemos especificar seletores $s_t \in S_t$,
compostos de $\overrightarrow{B_{s_t}} = [\beta_{s_t,1},\dots,\beta_{s_t,b-1}] 
\cup \overrightarrow{V_{s_t}} = [v_{s_t,1},\dots,v_{s_t,b-1}]$,
onde cada $\beta_{s_t,j} \in \overrightarrow{B_{s_t}}$ representa um parâmetro
de compilação ou configuração para a instância do problema, que assume o valor
especificado por $v_{s_t,i} \in \overrightarrow{V_{s_t}}$.
O seletor $s_t$ seleciona um valor para cada parâmetro de compilação e 
configuração definido para a instância:

\begin{equation*}
    SELECT_{\beta}(s_t, j) = v_{s_t,j} \in \overrightarrow{V_{s_t}} \quad | \quad \beta_{s_t,j} \in \overrightarrow{B_{s_t}},
    \quad\quad 1 \leq j < b
\end{equation*}

Com os seletores $s$ e $s_t$, podemos especificar um programa
$P_{s,s_t}$, e o objetivo de um \emph{autotuner} ideal é encontrar
os seletores $s^*,s_{t}^{*}$ que especifiquem o programa $P_{s^*,s_{t}^{*}}$, 
ou $P^*$, cujo tempo de execução $t^*$ --- representado por $T(P^*,H,n)$ --- 
é calculado executando-se o programa na arquitetura $H$, com entrada de tamanho
$n$, e deve satisfazer:

\begin{equation*}
    t^* = T(P^*,H,n) \leq T(P_{s^*,s_{t}^{*}},H,n), \quad \forall s \in S, \quad \forall s_t \in S_t
\end{equation*}

Em problemas computacionalmente fáceis, o tempo de execução dos programas
especificados pelos seletores será polinomialmente proporcional ao tamanho da 
entrada, facilitando o uso de testes empíricos para determinar as melhores
combinações algorítmicas. Problemas computacionalmente difíceis, por outro 
lado, precisam ser abordados de forma diferente --- através de, por exemplo,
aproximações e heurísticas --- cujo tempo de execução pode depender de 
diferentes características da entrada. Isso sugere que o melhor método de busca
no espaço de soluções para problemas computacionalmente difíceis será diferente
daquele para o caso de problemas computacionalmente fáceis.

O tempo de execução dos programas especificados pelos seletores também sofre
interferência de muitos outros fatores, como por exemplo outros processos 
executados pelo sistema operacional, que competem por recursos computacionais.
As propriedades da avaliação do desempenho de soluções para diferentes classes 
de problema contribuirão para o desenvolvimento e a escolha dos métodos
que serão utilizados no \emph{autotuner}.

Alguns detalhes do algoritmo para o \emph{autotuning} utilizado pelo PetaBricks
são discutidos na Seção \ref{sec:autotuner}, e a estratégia utilizada pelo AMF
é discutida na Seção \ref{sec:multsel}.

\subsection{PetaBricks} \label{sec:peta}

O sistema PetaBricks (\citet{ansel2009petabricks, ansel2014phd, 
ansel:xrds:2010, mitcsail-tr:2014})
consiste em uma linguagem, um compilador, e um ambiente de execução,
e seu objetivo é encontrar uma sequência de aplicação de algoritmos 
para a solução de um mesmo problema que especifique um programa com o melhor 
desempenho na arquitetura alvo, através de uma busca empírica num espaço
de soluções definido pelo programador.

A linguagem PetaBricks (\emph{.pbcc}) facilita a 
elaboração do espaço de busca pelo programador, que pode expor ao compilador 
--- através de construções de alto nível --- as escolhas algorítmicas 
para solução de um problema e suas ordens de execução válidas. O programa é 
compilado para a linguagem C++ e então passa pelo \emph{autotuner} 
(\citet{ansel2011efficient}), que utiliza um algoritmo genético para gerar e 
testar diferentes combinações, produzindo uma configuração otimizada para 
execução na arquitetura alvo.

\subsubsection{Linguagem}

A linguagem do sistema PetaBricks não possui mecanismos para controle do fluxo
de execução do programa. É possível especificar apenas modos de solução e 
condições de parada, e o fluxo de execução entre esses modos será determinado 
pelo processo de otimização.
Tomando como exemplo o problema da ordenação de vetores, um programa no
PetaBricks pode utilizar a primitiva \emph{either...or} 
(Figura~\ref{fig:sort2pbcc}) ou os conceitos de \emph{transform} e \emph{rule}
(Figura~\ref{fig:sortpbcc}) para determinar o espaço de algoritmos que podem 
ser combinados para gerar uma solução eficiente.

Uma \emph{transform} é análoga a uma função, contendo código que pode
ser chamado por outras \emph{transforms}, por código em outras linguagens
ou pela linha de comando. O cabeçalho de uma \emph{transform} pode ser composto
dos argumentos \emph{from}, \emph{to} e \emph{through}, que representam a
memória correspondente à entrada, saída e intermediários usados na 
\emph{transform}.
As possíveis escolhas de algoritmos feitas pelo programador são expostas 
em \emph{rules} dentro de uma \emph{transform}, que definem um algoritmo
para tratar determinada região de dados. O cabeçalho de uma \emph{rule} pode
conter argumentos para variáveis de entrada, saída, e possíveis restrições sua 
para aplicação, codificados pelas primitivas \emph{from}, \emph{to} e 
\emph{where}.

Na \emph{transform} SortSubArray, no código da Figura~\ref{fig:sortpbcc}, são 
definidas diversas \emph{rules} para a ordenação de subvetores, que
são utilizadas pela \emph{transform} Sort para ordenar um vetor. Após a
compilação, o resultado é um programa que executa uma combinação dos dois
métodos de ordenação implementados, de acordo com o tamanho do subvetor a
ser ordenado, e dependente da máquina onde foi compilado. No entanto, esperamos
que o programa execute a regra QuickSort para subvetores grandes, e a 
regra InsertionSort quando o tamanho for menor, por conta de características
inerentes a esses algoritmos.

A primitiva \emph{either...or} permite definir explicitamente um espaço
de algoritmos, e cabe ao \emph{autotuner} selecionar uma dentre as cláusulas
do \emph{either...or} a cada vez que a primitiva é executada.

\begin{figure}[H]
    \centering
    \lstinputlisting{./code/Sort.pbcc}
    \caption{Sort.pbcc}
    \label{fig:sortpbcc}
\end{figure}

\begin{figure}[H]
    \centering
    \lstinputlisting{./code/Sort2.pbcc}
    \caption{Sort2.pbcc}
    \label{fig:sort2pbcc}
\end{figure}

\subsubsection{Compilador}

\begin{figure}[H]
    \begin{tikzpicture}
        [node distance=.76cm,
        start chain=going right]        
        \footnotesize
            \node[fillchain, join] (compiler) {Compilador};
            \begin{scope}[start branch=peta]
                \node[punktchain, on chain=going above] (pbcc) {Código \emph{.pbcc}};
            \end{scope}
            \node[punktchain, join] (cpp) {Código C++};
            \node[punktchain, join] (bin) {Binário};
            \begin{scope}[start branch=tuner]
                \node[fillchain, on chain=going above] (auto) {Autotuner};
            \end{scope}
            \node[punktchain, join] (exe) {Execução};
            \begin{scope}[start branch=conf]
                \node[punktchain, on chain=going above] (arqconf) {Arquivo de Configuração};
            \end{scope}
            \begin{scope}[start branch=env]
                \node[fillchain, on chain=going below] (envexe) {Ambiente de Execução};
            \end{scope}
            \draw[|-,-|,->, thick,] (arqconf.south) |-+(0,-1em)-| (exe.north);
            \draw[|-,-|,->, thick,] (envexe.north) |-+(0,0)-| (exe.south);
            \draw[|-,-|,<->, thick,] (bin.north) |-+(0,2em)-| (auto.south);
            \draw[|-,-|,<->, thick,] (auto.north) |-+(0,2em)-| (arqconf.north);
            \draw[|-,-|,->, thick,] (pbcc.south) |-+(0,0)-| (compiler.north);
    \end{tikzpicture}
    \centering
    \caption{Estrutura do Sistema PetaBricks}
    \label{fig:petabricks-overview}
\end{figure}

O compilador do PetaBricks contém três componentes principais: um 
compilador fonte-para-fonte da linguagem PetaBricks para C++; um 
\emph{autotuner}; e uma biblioteca de execução utilizada pelo código gerado.
Esses componentes permitem ao sistema PetaBricks gerar um programa final 
otimizado para a arquitetura alvo, que utiliza os algoritmos especificados pelo
programador numa combinação que minimiza o tempo de execução nesta arquitetura.
A Figura~\ref{fig:petabricks-overview} ilustra as relações entre esses 
componentes e os dados que utilizam no processo de compilação.

A primeira tarefa do compilador é fazer a análise sintática do código fonte, 
determinando as regiões --- intervalos no tamanho da entrada, por exemplo ---
onde cada \emph{rule} pode ser aplicada. Essas regiões são calculadas à partir
de inferências sobre o código fonte, e podem ser definidas pelo programador
através da primitiva \emph{where}. Com essa informação o compilador pode 
construir um grafo de dependências para as escolhas algorítmicas,
que é codificado no programa em C++ gerado, e será utilizado pelo ambiente de
execução e \emph{autotuner} do PetaBricks, explorando as escolhas disponíveis
e otimizando o programa.

No próximo passo, diferentes versões do programa serão geradas e executadas, 
contendo combinações de parâmetros de compilação e algoritmos diferentes, 
produzindo arquivos de configuração que podem servir de parâmetro para a 
continuação do processo de busca ou para gerar um arquivo executável final, 
já otimizado.

\subsubsection{Autotuner}\label{sec:autotuner}

O \emph{autotuner} do PetaBricks realiza a busca pela melhor solução de 
determinado problema, representada por uma combinação dos algoritmos 
especificados pelo programador. O \emph{INCREA}~\citet{ansel2011efficient} é o 
algoritmo genético responsável por gerar e avaliar as possíveis soluções 
no \emph{autotuner}, e faz isso de maneira incremental, começando
o processo de seleção sobre codificações mais simples de solução e utilizando 
os resultados obtidos para orientar a busca sobre codificações mais complexas.

A estratégia do \emph{INCREA} funciona bem quando o tamanho da codificação
implica num alto custo de avaliação da solução mas, ao mesmo tempo, soluções
melhores têm um custo menor. Isso acontece pois o algoritmo começa utilizando
problemas de tamanho pequeno e aumenta o tamanho da codificação utilizada na 
seleção conforme encontra boas soluções, garantindo que as maiores codificações
encontradas serão próximas a candidatos com boa avaliação e portanto terão um
baixo custo de avaliação. Outra vantagem do algoritmo é ter bons resultados
quando o processo de avaliação apresenta ruído ou alta variação nas medidas.
Em contrapartida, o algoritmo não tem bons resultados quando o espaço de busca
contém muitos pontos de máximo locais --- podendo impedir a 
exploração de outras áreas do espaço de busca --- ou as soluções de 
codificações menores não são reutilizáveis para soluções maiores.

O genoma que codifica as soluções tem tamanho $(2m + 1)k + n$,
onde $k$ representa o número de seletores, $m$ os
limites dos intervalos para cada algoritmo e $n$ o número de \emph{tunables} 
utilizados na seleção. O número de pares de algoritmos e limites de intervalo
ativos no genoma começa em $1$ e é incrementado a cada geração.
A avaliação do genoma consiste em executar o programa gerado pela
configuração representada no genoma, o valor do \emph{fitness} é o inverso
do tempo de execução. O tamanho da entrada testada na execução do programa
começa em $1$ e dobra a cada geração.

\subsection{Adaptive Multi-Selection Framework} \label{sec:amf}

O \emph{Adaptive Multi-Selection Framework}, ou AMF 
(\citet{goldman2012framework}), permite resolver instâncias de um problema 
através de combinações de algoritmos para solução dessas instâncias. 
Os algoritmos a ser combinados e as instâncias a ser resolvidas são
especificados numa aplicação do AMF, que pode também fornecer dados sobre
o desempenho dos algoritmos na arquitetura alvo do programa final.
Através da técnica da Multi-Seleção, o AMF gera uma divisão dos recursos 
computacionais disponíveis --- por exemplo, núcleos do processador --- entre 
esses algoritmos, de acordo com seu desempenho na solução das instâncias 
apresentadas. Para isso, utiliza heurísticas para a aproximação do problema do
escalonamento da distribuição de recursos, assumindo que
um recurso não pode ser dividido entre algoritmos, e que nenhum recurso 
deixará de ser utilizado --- o problema, abreviado como dRSSP (\emph{Discrete 
Resource Sharing Scheduling Problem}), é descrito e estudado por 
\citet{bougeret2009combining,bougeret2011approximating}. Finalmente,
a divisão de recursos selecionada é executada como um algoritmo 
\emph{portfolio} (\citet{huberman1997economics}).

O AMF é implementado em C/C++ e seu componente central é o 
\textbf{AMF\_Learner}, que obtém informações sobre a arquitetura
alvo, serve de base de conhecimentos para os demais componentes,
e gera a combinação de algoritmos para as aplicações registradas no
arcabouço. A resolução de uma instância através da aplicação é coordenada pelo 
componente \textbf{AMF\_Solver} que, com as otimizações para distribuição de 
recursos calculadas pelo componente \textbf{AMF\_Optimizer}, executa o 
algoritmo \emph{portfolio} através do componente \textbf{AMF\_Portfolio}.
Os componentes \textbf{AMF\_AContainer} e \textbf{AMF\_BContainer} especificam
os algoritmos disponíveis e as avaliações de desempenho sobre as instâncias
utilizadas na otimização da distribuição de recursos, respectivamente. A Figura
\ref{fig:amf-overview} ilustra as relações entre os componentes do AMF.

\begin{figure}[H]
    \begin{tikzpicture}
        [node distance=.5cm,
        start chain=going right]
        \footnotesize
            \node[punktchain] (port) {AMF\_Portfolio};
            \begin{scope}[start branch=tuner]
                \node[punktchain, on chain=going below] (alg) {AMF\_AContainer};
            \end{scope}
            \node[fillchain] (learner) {AMF\_Learner};
            \begin{scope}[start branch=solv]
                \node[punktchain, on chain=going above] (solver) {AMF\_Solver};
            \end{scope}
            \node[punktchain] (opt) {AMF\_Optimizer};
            \begin{scope}[start branch=tuner]
                \node[punktchain, on chain=going below] (bench) {AMF\_BContainer};
            \end{scope}
            \draw[|-,-|,-, thick,] (port.east) |-+(0,0)-| (learner.west);
            \draw[|-,-|,-, thick,] (learner.east) |-+(0,0)-| (opt.west);
            \draw[|-,-|,-, thick,] (learner.north) |-+(0,0)-| (solver.south);
            \draw[|-,-|,-, thick,] (port.north) |-+(0,1)-| (solver.west);
            \draw[|-,-|,-, thick,] (opt.north) |-+(0,1)-| (solver.east);
            \draw[|-,-|,-, thick,] (port.south) |-+(0,0)-| (alg.north);
            \draw[|-,-|,-, thick,] (alg.east) |-+(0,0)-| (learner.south);
            \draw[|-,-|,-, thick,] (bench.west) |-+(0,0)-| (learner.south);
    \end{tikzpicture}
    \centering
    \caption{Componentes do AMF e suas relações.}
    \label{fig:amf-overview}
\end{figure}

\subsubsection{Multi-Seleção} \label{sec:multsel}

A técnica da Multi-Seleção no AMF utiliza três fases para resolver determinada 
instância ou grupo de instâncias de um problema. Primeiro, o AMF seleciona
algoritmos candidatos para a instância, que podem ser paralelos ou sequenciais.
Depois, calcula a distribuição de recursos para os algoritmos selecionados, 
baseando-se nos resultados de desempenho dos algoritmos na instância. 
Finalmente, os algoritmos selecionados são executados concorrentemente, 
seguindo a distribuição calculada na segunda fase, e quando qualquer um
deles resolver a instância, os outros são interrompidos e a execução da
aplicação do AMF termina. A Multi-Seleção no AMF pode ser descrita da
seguinte forma (\citet{goldman2012framework}):

Considere o conjunto $I$ de instâncias de um problema, e as instâncias $i_k \in
I, \quad 0 \leq k < n$, onde $n$ é o número de instâncias especificadas. 
Considere também o conjunto $A$, de algoritmos candidatos para solução da
instância $i_k$. O AMF seleciona um conjunto $A_{i_k} \subseteq A$ para
execução em \emph{portfolio}, que levará um tempo total $T(i_k) = 
T_s(i_k) + T_{rs}(i_k) + T_{ep}(i_k)$ para resolver a instância $i_k$,
onde $T_s$, $T_{rs}$ e $T_{ep}$ calculam o tempo para fazer a seleção
dos algoritmos, distribuir os recursos e executar o \emph{portfolio},
respectivamente.

O AMF pode operar em dois modos de seleção. O modo \emph{offline}
seleciona o conjunto $A_{i} = A, \forall i \in I$, fazendo com que
$T_s(i) + T_{rs}(i)$ seja constante, e com que $T(i)$ dependa mais
do tempo de execução do \emph{portfolio}. No modo \emph{online},
a seleção do conjunto $A_{i_k} \subseteq A$ tem como objetivo minimizar
$T_{ep}(i_k)$ através da seleção de algoritmos mais eficientes, mesmo
que isso aumente $T_s(i) + T_{rs}(i)$.

\subsubsection{Distribuição de Recursos} \label{sec:drssp}

Depois de selecionados os algoritmos para solução de uma dada instância $i$,
é feito o cálculo da distribuição dos recursos computacionais disponíveis entre
os algoritmos. A distribuição é calculada por aproximações de soluções para
o dRSSP, um problema NP-difícil. \citet{bougeret2011approximating} prova a 
complexidade do dRSSP, e descreve o problema da seguinte forma:

Considere o conjunto $A = \{\alpha_1, \dots, \alpha_k\}$ de $k$ algoritmos que
resolvem instâncias de um determinado problema, o conjunto $I = \{i_1, \dots, 
i_n\}$ de $n$ instâncias do problema, e o conjunto $R = \{0, \dots, m\}$ com
$m$ recursos computacionais idênticos, onde pode-se executar algoritmos de $A$
em instâncias $i \in I$. Define-se $C(\alpha_i,i_j,r)$, o custo
de um algoritmo $\alpha_i \in A$ na instância $i_j \in I$, com $r \in R$ 
recursos. O custo, neste caso, é o tempo de execução do 
algoritmo na resolução da instância.

Calcula-se então, através das heurísticas para o dRSSP implementadas no AMF,
um \emph{portfolio} (\citet{huberman1997economics}),
que consiste num vetor $\overrightarrow{S} = [s_1, \dots, s_k] \quad | \quad 
s_i \in R, \quad \sum_{} s_i \leq m$, onde $s_i \in \overrightarrow{S}$ 
determina o número de recursos computacionais alocados para execução
do algoritmo $\alpha_i \in A$. Como a execução do algoritmo 
\emph{portfolio} termina quando qualquer um dos algoritmos encontra
uma solução para a instância, o custo para qualquer instância
$i_j \in I$, com alguma distribuição de recursos $\overrightarrow{S_{i_j}}$,
pode ser escrito como $C(\overrightarrow{S_{i_j}}, i_j) = 
min \{C(\alpha_i,i_j,s_i)\}|_{\alpha_i \in A, s_i > 0}$. O objetivo
então é calcular a distribuição de recursos $\overrightarrow{S^*}$ 
que, para a solução de múltiplas instâncias, satisfaça:

\begin{equation*}
\sum_{i_j \in I} C(\overrightarrow{S^*},i_j) \quad \leq \quad \sum_{i_j \in I} 
C(\overrightarrow{S_k}, i_j), \quad \forall \overrightarrow{S_k}
\end{equation*}

Alternativamente, quando o objetivo for resolver uma instância específica e 
não instâncias em blocos, pode-se calcular um $\overrightarrow{S^*}$ que
satisfaça:

\begin{equation*}
    max \{C(\overrightarrow{S^*},i_j)\} |_{i_j \in I} \quad \leq \quad
    max \{C(\overrightarrow{S_k}, i_j)\} |_{i_j \in I}, \quad \forall 
    \overrightarrow{S_k}
\end{equation*}

\subsubsection{Algoritmo Portfolio} \label{sec:portfolio}

Após a seleção dos algoritmos para solução da instância e do
cálculo de uma distribuição de recursos entre eles, os algoritmos
são executados em paralelo na arquitetura alvo, seguindo a distribuição
de recursos calculada. Quando o primeiro algoritmo no \emph{portfolio}
encontra uma solução para a instância, os demais algoritmos em execução 
são interrompidos pelo AMF e a instância está resolvida.

\section{Trabalhos Relacionados} \label{sec:trabrel}

Nas últimas três décadas, muitos trabalhos apresentaram sistemas e técnicas 
para a otimização automática empírica, permitindo a construção de programas e 
bibliotecas de eficiência portável em diferentes domínios de problemas.
\citet{bilmes1997} apresentam em 1997 o sistema PHiPAC para otimização
automática, dependente da arquitetura, da multiplicação de matrizes. O PHiPAC
utiliza \emph{scripts} de busca, que geram, compilam e testam os programas
candidatos, para encontrar a melhor combinação de parâmetros para geração do 
código de multiplicação de matrizes. Em retrospectiva recente, 
\citet{bilmes2014phipacretro} explicam como o código gerado pelo PHiPAC atingiu
desempenho comparável à das bibliotecas para multiplicação de matrizes 
otimizadas pelos fabricantes de computadores na época, motivando o uso de
técnicas de otimização automática em diferentes áreas 
(\citet{demmel2009accelerating}).

Foram desenvolvidos vários sistemas para construção e otimização de bibliotecas
para problemas em domínios específicos. Entre eles, o sistema ATLAS 
(\citet{whaley1998, whaley2005}) utiliza técnicas empíricas para selecionar
algoritmos para multiplicação de matrizes. O projeto UHFFT 
(\citet{ali2007scheduling}) otimiza o cálculo da transformada rápida
de Fourier em arquiteturas \emph{multicore}. O sistema SPIRAL 
(\citet{puschel2005spiral}) otimiza cálculos de processamento de sinais
digitais.

Outros trabalhos propõe abstrações em alto nível --- linguagens e arcabouços 
--- facilitando o processo de especificação dos espaços de busca para a 
otimização e a implementação de programas adaptáveis. O STAPL 
(\citet{thomas2005stapl}) é uma biblioteca de \emph{templates} para C++
que permite o uso de algoritmos adaptativos. A representação intermediária
INSPIRE (\citet{jordan2013inspire}) unifica construções de linguagens e 
especificações para programação paralela, como OpenCL, OpenMP e MPI.
\citet{yu2004adaptive} apresentam um arcabouço para geração \emph{off-line}
de mecanismos de seleção algorítmica.

Alguns trabalhos discutem os processos e os métodos de busca no espaço
de soluções especificadas para um problema. Entre eles,
\citet{yotov2003} comparam a otimização por busca empírica e por 
modelagem, apresentando um modelo com desempenho comparável à busca empírica.
\citet{vuduc2004} apresentam e comparam alguns
modelos estatísticos que auxiliam no processo de busca empírica.
\citet{cooper2002compilation} discutem a importância da
ordem em que as otimizações de compilação são aplicadas, e apresentam
um protótipo para seleção e adaptação dessa ordem.
Em \citet{cooper2002adaptive}, defende-se o uso de métodos de busca
por seleção aleatória e descreve-se um arcabouço para implementação
de compiladores adaptativos.

\section{Objetivo} \label{sec:obj}

O objetivo deste projeto de pesquisa é implementar as estratégias para o 
cálculo de distribuições de recursos computacionais e execução de algoritmos 
\emph{portfolio}, utilizadas no AMF (\citet{goldman2012framework}), como método
para o \emph{autotuning} de problemas computacionalmente difíceis --- 
NP-completos, por exemplo --- no sistema PetaBricks (\citet{ansel2014phd}).
Essas estratégias devem substituir o \emph{autotuner} do PetaBricks quando o 
problema a ser resolvido for computacionalmente difícil, calculando uma
distribuição de recursos computacionais para as possibilidades algorítmicas 
especificadas num programa na linguagem PetaBricks. As estratégias devem também
implementar um ambiente de execução que dê suporte à execução de um 
\emph{portfolio} com a distribuição calculada.

\section{Metodologia} \label{sec:met}

\section{Cronograma} \label{sec:sched}

A figura \ref{fig:sched} apresenta o cronograma das atividades que serão
desenvolvidas durante o projeto.

\begin{figure}[H]
    \centering
\begin{center}
    \begin{tabular}{ | >{\centering\arraybackslash}p{3cm} | 
    >{\centering\arraybackslash}p{3cm} | 
    >{\centering\arraybackslash}p{3cm} | 
    >{\centering\arraybackslash}p{3cm} |}
    \multicolumn{4}{c}{Semestres} \\
    \hline
    1º & 2º & 3º & 4º \\ \hline 
    \cellcolor{gray!94} Revisão Bibliográfica & \multicolumn{3}{c|}{} \\ 
    \cellcolor{gray!70} Estudo de Código do PetaBricks & \multicolumn{3}{c|}{} \\ 
    \cellcolor{gray!56} Estudo de Código do AMF & \multicolumn{3}{c|}{} \\ 
      & \multicolumn{2}{l}{\cellcolor{gray!56} Implementação} &   \\ 
      & \multicolumn{3}{l|}{\cellcolor{gray!70} Testes de Funcionalidade} \\ 
      & \multicolumn{3}{l|}{\cellcolor{gray!94} Aplicação dos Benchmarks, Testes de Desempenho} \\ 
    \hline
    \end{tabular}
\end{center}
    \caption{Cronograma para as atividades do projeto.}
    \label{fig:sched}
\end{figure}

No primeiro semestre, serão feitos um estudo da bibliografia da área de 
otimização automática, e estudos dos detalhes de implementação dos sistemas
PetaBricks e AMF.

Nos semestres subsequentes será iniciado o ciclo de desenvolvimento da
proposta, integrando as estratégias para cálculo de distribuição de recursos
ao autotuner do PetaBricks. O ciclo compreenderá a implementação, testes
de funcionalidade e desempenho, utilizando os \textit{benchmarks} especificados
na Seção \ref{sec:met}.

\newpage
\bibliographystyle{plainnat}
\bibliography{projeto}

\end{document}
